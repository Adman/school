\documentclass[12pt]{article}

\usepackage{amsmath}
\usepackage{amssymb}
\usepackage{cmap}
\usepackage[slovak]{babel}
\usepackage[utf8]{inputenc}
\usepackage[T1]{fontenc}

\newcommand{\ud}{\;\mathrm{d}}
\newcommand{\cinterval}[1]{\ensuremath{\left[#1\right]}}
\newcommand{\ointerval}[1]{\ensuremath{\left(#1\right)}}
\newcommand\tab[1][0.5cm]{\hspace*{#1}}


\begin{document}

Pre $\alpha: \cinterval{a,b} \rightarrow \mathbb{R}$ takú, že
$V\alpha \ointerval{a,b} < \infty$ definujeme
$\sum\limits_{i=0}^{n-1} f(c_i) \Delta \alpha$, kde $c_i \in \mathrm{I}_i$,
$f$ je $\mathrm{R}-\mathrm{S}$ integrovateľná vzhľadom k $\alpha$, ak \newline
\tab $\exists A\in \mathbb{R} \ \ \forall \varepsilon > 0 \ \ \exists \delta > 0$
$\exists D_{\cinterval{a,b}} \ \ \mu(D)<\delta \ \ \forall c_i \in \mathrm{I}_i$
\ $\Big|\sum\limits_{i=0}^{n-1} f(c_i)\Delta\alpha_i-A\Big| < \varepsilon$ \newline
v takom prípade definujeme $\int_{a}^{b}f\ud x = A$.

\begin{enumerate}
    \item Ukážte, že ak $\alpha: \cinterval{a,b}\rightarrow\mathbb{R}$ je rastúca,
        tak tieto dve definície sú ekvivalentné.

    \item Dokážte, že pre $f,\alpha: \cinterval{a,b}\rightarrow\mathbb{R}$,
        $f\in\mathcal{R}(\alpha)$, platí: \newline
        \begin{enumerate}
            \item $\int_a^b cf\ud\alpha = c\int_a^b f\ud\alpha,\tab c\in \mathbb{R}$
            \item $\int_a^b f\ud(c\alpha) = c\int_a^b f\ud\alpha,\tab c\in \mathbb{R}$
            \item $\int_a^b (f+g)\ud\alpha = \int_a^b f\ud\alpha + \int_a^b g\ud\alpha$,
                \tab teda ak $f,g \in \mathcal{R}(\alpha)$, potom $f+g \in \mathcal{R}(\alpha)$
                a platí vzťah
            \item $\int_a^b f\ud(\alpha + \beta) = \int_a^b f\ud\alpha + \int_a^b f\ud\beta$
                \tab $f \in \mathcal{R}(\alpha)\cap\mathcal{R}(\beta)$, potom $f\in \mathcal{R}(\alpha + \beta)$
        \end{enumerate}

    \item $f,\alpha: \cinterval{a,b}\rightarrow\mathbb{R}$, $f$ je spojitá a $V\alpha<\infty$,
        potom $f\in\mathcal{R}(\alpha)$

    \item $f$ spojitá na $\cinterval{a,b}$, $s\in\left(a,b\right]$ a označme
        $\chi_{\cinterval{s,b}}=\alpha(X)$. Potom
        $\int_a^b f\ud\alpha = f(s)$\tab (v prípade $s=a$ zvolíme
        $\alpha(X)=\chi_{\left(a,b\right]}$)

    \item $\alpha:\cinterval{a,b}\rightarrow\mathbb{R}$ rastúca, spojitá v
        $x_0 \in \cinterval{a,b}$, $f: \cinterval{a,b}\rightarrow\mathbb{R}$,\newline
        $f(x)=\begin{cases}1,&x=x_0,\\0,&\textrm{inak}\end{cases}$.
        Potom $f \in \mathcal{R}(\alpha)$, $\int_a^b f\ud\alpha=0$

    \item $f:\cinterval{a,b}\rightarrow\mathbb{R}$ diferencovateľná, $f^{\prime}\in\mathcal{R}$.
        Potom $Vf \left(a,b\right) < \infty$ a $\int_a^b |f^{\prime}(x)|\ud x=Vf$

\end{enumerate}

\end{document}

